\documentclass{article}
\usepackage[margin=1.2in]{geometry}
\usepackage{datetime}
%\usepackage{hyperref}      %for \url macro
\usepackage{microtype}     %attempt to fix issue with justification protrusion (in references)
\usepackage{amssymb}       % for formatting less/greater than symbols
\usepackage{amsmath}
\usepackage{enumitem}      %for changing spacing in bulleted lists
\usepackage{subfigure}        %for subfigures


\renewcommand{\arraystretch}{1.25}

\usepackage[gobble=auto, runall=true]{pythontex}
\usepackage{float} %for forcing position of images

\usepackage{graphicx}
\graphicspath{ {../images/} }
\usepackage[export]{adjustbox}
\usepackage[justification=centering]{caption}

\usepackage{listings}   %for typesetting code
\usepackage{color}
\definecolor{codegreen}{rgb}{0,0.6,0}
\definecolor{codegray}{rgb}{0.5,0.5,0.5}
\definecolor{codepurple}{rgb}{0.58,0,0.82}
\definecolor{backcolour}{rgb}{0.95,0.95,0.92}
\lstdefinestyle{mystyle}{
    backgroundcolor=\color{backcolour},
    commentstyle=\color{codegreen},
    keywordstyle=\color{codepurple},
    numberstyle=\tiny\color{codegray},
    stringstyle=\color{codepurple},
    basicstyle=\footnotesize,
    breakatwhitespace=false,
    breaklines=true,
    captionpos=b,
    keepspaces=true,
    %numbers=left,
    numbersep=5pt,
    showspaces=false,
    showstringspaces=false,
    showtabs=false,
    tabsize=2
}
\lstset{style=mystyle}

\frenchspacing                   %removes extra spacing after a period at the end of a sentence.
\newdateformat{daymonthyear}{\THEDAY\ \monthname\ \THEYEAR}

\title{CSC411 Machine Learning \\ Project 3: Fake News}
\author{ Ariel Kelman \\ Student No: 1000561368
         \\ \\
         Gideon Blinick \\ Student No: 999763000 }
\daymonthyear



\begin{document}
   \maketitle{}


   \section{Dataset Description}

Both datasets (real and fake) contain headlines about U.S. President Donald Trump. All characters are lowercase.  
It is certainly feasible to determine whether a headline is real or fake news based on the precense or absence of certain keywords in the headlines.
For the part, we analyzed the data in 2 ways:
\begin{itemize}
  \item  First we looked at the words that had the largest difference in percentage for appearence in the datasets. For example,
the words "the", "trump" and "hillary" appeared more often in the fake news headlines by percentage by 20.0\%, 10.4\% and 10.3\%, respectively, while the words
"donald", "trumps", and "us" appeared more often in the real news headlines by 24.5\%, 10.8\%, and 8.8\%, respectively. 
  \item Second, we looked at words that only appear in one dataset and not the other by count. Count was neccessary here, because percentage difference cannot apply.
So the top words in the fake dataset that did not appear in the real dataset were "breaking" (27 headlines), "u" (19), and "soros" (18), while the top words in the real
dataset that were not in the fake one were "korea" (79), "turnbull" (55), and "travel" (52). These words are especially relevant because their precense in a headline
allows us to automatically classify the headline as fake or real.
\end{itemize}




   %\subsection{Results \& Report Reproducibility}



   \section{Naive Bayes Implementation}




   \section{Predictive Factors}




   \section{Logistic Regression}



   \section{Logistic Regression vs. Naive Bayes}

  \section{Miscle}
   
\section{Decision Tree}


   \section{Decision Tree - Information Theory}




\end{document}
