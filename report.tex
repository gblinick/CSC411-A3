\documentclass{article}
\usepackage[margin=1.2in]{geometry}
\usepackage{datetime}
%\usepackage{hyperref}      %for \url macro
\usepackage{microtype}     %attempt to fix issue with justification protrusion (in references)
\usepackage{amssymb}       % for formatting less/greater than symbols
\usepackage{amsmath}
\usepackage{enumitem}      %for changing spacing in bulleted lists
\usepackage{subfigure}        %for subfigures


\renewcommand{\arraystretch}{1.25}

\usepackage[gobble=auto, runall=true]{pythontex}
\usepackage{float} %for forcing position of images

\usepackage{graphicx}
\graphicspath{ {../images/} }
\usepackage[export]{adjustbox}
\usepackage[justification=centering]{caption}

\usepackage{listings}   %for typesetting code
\usepackage{color}
\definecolor{codegreen}{rgb}{0,0.6,0}
\definecolor{codegray}{rgb}{0.5,0.5,0.5}
\definecolor{codepurple}{rgb}{0.58,0,0.82}
\definecolor{backcolour}{rgb}{0.95,0.95,0.92}
\lstdefinestyle{mystyle}{
    backgroundcolor=\color{backcolour},
    commentstyle=\color{codegreen},
    keywordstyle=\color{codepurple},
    numberstyle=\tiny\color{codegray},
    stringstyle=\color{codepurple},
    basicstyle=\footnotesize,
    breakatwhitespace=false,
    breaklines=true,
    captionpos=b,
    keepspaces=true,
    %numbers=left,
    numbersep=5pt,
    showspaces=false,
    showstringspaces=false,
    showtabs=false,
    tabsize=2
}
\lstset{style=mystyle}

\frenchspacing                   %removes extra spacing after a period at the end of a sentence.
\newdateformat{daymonthyear}{\THEDAY\ \monthname\ \THEYEAR}

\title{CSC411 Machine Learning \\ Project 3: Fake News}
\author{ Ariel Kelman \\ Student No: 1000561368
         \\ \\
         Gideon Blinick \\ Student No: 999763000 }
\daymonthyear



\begin{document}
   \maketitle{}


   \section{Introduction}
   ...

   %\subsection{Results \& Report Reproducibility}
   All results and plots can be generated using the appropriate python files.
   All code is python 3, run with Anaconda 3.6.3.
   Code can be found in \texttt{digits.py}.
   Running the code will save all generated images in the \texttt{resources} folder,
   where they are used by \LaTeX. Note that some of the sections require the code for
   other sections (in the same file) to be run first. Additionally, the files for parts 8 and 10
   contain similar functions; it may be useful to restart the python shell when switching files
   to ensure reproducibility.
   To reproduce the report, simply run it through \LaTeX. This will pull the most recently
   generated figures from the \texttt{resources} folder.


   \section{Naive Bayes Implementation}




   \section{Predictive Factors}



   \subsection{Logistic Regression}



   \section{Comparison of Models}



   \section{Analyzing Logistic Regression}


   \section{Decision Tree}


   \section{Decision Tree - Information Theory}




\end{document}
